\chapter{Naissance d'E-protect :}

La création d'E-protect nous est venue lors d'un séminaire au Danemark dans la ville d'Aalborg.\\
Nous étions, le reste de l'équipe et moi-même, enivrés dans la folie des nuits danoises, en train d'écumer les bars et pistes de danse de la vielle ville. Après avoir revu l'ensemble de nos pas, nous décidions d'aller à la rencontre des locaux, c'est à ce moment-là que nous rencontrions Torsten. Le regard vide au milieu d'une bande d'amis bien destinés à lui remonter le moral. Une mission qui après quelque verres nous convenait parfaitement.\\
Nous nous assayâmes un instant autour de sa table et avant même que nous ayons le temps d'entamer les présentations, il coupa "ce soir je bois pour oublier". Oublier quoi ? Oublier le fait qu'il venait de se faire cambrioler et qu'en attendant les résultats de l'enquête il noyait son dégoût dans des boissons énergétiques.\\

Le lendemain, nous reprenions notre séminaire et la création d'application androïde et IOS.\\

C'est alors apparu comme une évidence : Torsten, les applications, E-protect était né.\\

Nous commencions alors nos recherches. En 2013 il y a eu 382 000 cambriolages et 263 000 incendies domestiques en France. L'importance de développer des solutions innovantes de protections pour lutter contre ces faits est bien réelle.\\

Pour information, de nos jours l'achat d'une alarme pour le domicile, entraîne l'intervention d'un technicien et si nous souhaitons apporter des modifications à l'installation un deuxième rendez-vous doit être pris. Perte de temps, surplus financier, le client est dépendant du bon vouloir de l'artisanat.\\

Il s'agit ici d'une intervention sérieuse, nécessitant une forte part de confiance en l'installateur, un inconnu représentant un risque potentiel.\\

Voici ce que nous proposons et vous recommandons : Un système d'alarme modulable et installable par le client. Une avancé remarquable dans le domaine de la protection. Avec ses caméras destinées à la prévention et une nouvelle génération de capteurs invisibles révisés par un design innovant, le client pourra imaginer avec l'aide de nos conseillers son installation.\\

Notre offre se décompose en deux parties:\\
Une partie Hardware avec une centrale d'alarme, sous forme de cadre numérique munie d'une batterie, connecté au réseau EDF et munie d'un système BACKUP (qui permet d'assurer le fonctionnement de l'alarme le temps de prévenir l'utilisateur via SIGFOX) destinée à gérer un parc de capteurs connectés suivant une topologie meshé (ZigBee/Bluetooth Low Energy/CSRmesh). Celle-ci sera connecté à internet via Ethernet ou WIFI et sera capable de communiquer via un protocole de communication sans fil (ex: SIGFOX).\\
L'alarme interagira aussi avec une partie software (Sigfox/Ethernet). Une application mobile et un site web permettront de se tenir informé de l'état de fonctionnement de l'installation. De plus et cela à n'importe quelle moment et où qu'il soit, le client pourra configurer l'alarme à distance via application Android/iOS/Windows Phone (Javascript/PHP/C\#/SQL).