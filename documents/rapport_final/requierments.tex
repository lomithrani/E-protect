\chapter{Requierments / Opportunity :}

En tête du classement des regions les plus touchées par les infractions de domiciles nous retrouvons la Guadeloupe. Avec un taux de 6.5 cambriolages pour 1000 habitants, elle précède le Vaucluse (6.4 pour 1000 habitants). De manière générale, les départements d’outre-mer sont parmi les plus exposés au risque de cambriolage. La Guyane affiche ainsi une statistique de 6 pour 1000 habitants. Concernant le France métropolitaine, le sud-est est particulièrement touché par les affaires de cambriolages d’habitations principales.\\

Six départements ont un taux compris entre 5.4 et 6.5 pour 1000 habitants (les Pyrénées-Orientales, l’Hérault, le Gard, le Vaucluse, les Bouches du Rhône et les Alpes-Maritimes).\\

L’Île de France que l’on pourrait considérer comme une région très exposée limite la casse et reste dans la moyenne nationale de 2.7/1000 habitants. A l’inverse, le risque de se faire cambrioler est extrêmement faible dans le centre de la France. Les résidents du Cantal peuvent dormir sur leurs deux oreilles avec seulement 0.3 cambriolage pour 1000 habitants département du Cantal. La Haute-Corse connaît la plus forte variation à la hausse avec une augmentation de $71.4\%$.\cite{www:ondrp}\\
