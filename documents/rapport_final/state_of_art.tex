\chapter{Etat de l'art :}

\section{Alarme en France :}

\textbf{Diagral}\\
\begin{figure}[h!]
	\begin{center}
		\includegraphics[width=0.4\linewidth]{../images/diagral}
	\end{center}
	\caption{\cite{www:diagral}}
\end{figure}
Diagral, produit de la société Atral, créée en 1985, appartient aujourd'hui au groupe Hager.\\
Le système Diargal s'appuit sur des alarmes à transmission radio.\\
Note obtenue dans le comparatif Alarmes de 60 millions de consommateurs : 15.5/20 (site Français spécialiste du marché des alarmes domestiques)\\
\textbf{Définition du produit :}
\begin{itemize}
\item Centrale d'alarme DIAG90AGFK
\end{itemize}

\textbf{Différences remarquées en comparaison avec le système E-protect :}
\begin{itemize}
\item Sirène intégrée
\item Annonces vocales
\item Division du batiment en zone (de 1 à 4) pour adapter les zones à sécuriser
\item Communication d'alerte par téléphone en option, via un module de transmission RTC ou GSM\\
\end{itemize}


\textbf{Delta dore}\\
\begin{figure}[h!]
	\begin{center}
		\includegraphics[width=0.4\linewidth]{../images/Deltadore}
	\end{center}
	\caption{\cite{www:deltadore}}
\end{figure}

Créée dans les années 70, Delta Dore est une société spécialisé dans les économies d'énergie avant d'élargir son secteur d'activité en intégrant le marché des alarmes sans fils en 2000.\\
Note obtenue dans le comparatif Alarmes de 60 millions de consommateurs : 12/20 (site Français spécialiste du marché des alarmes domestiques)\\

\textbf{Définition du produit :}
\begin{itemize}
\item Centrale d'alarme sans fil à transmission radio 868 MHz
\end{itemize}

\textbf{Différences remarquées en comparaison avec le système E-protect :}
\begin{itemize}
\item Sirène intégrée
\item Fonctionne sur pile pour éviter les défaut d'alimentation
\item Détecteur d'ouverture MICRO COX alimenté par pile
\item Module de transmission RTC ou GSM en option\\
\end{itemize}

\textbf{Tag Technologies}\\
\begin{figure}[h!]
	\begin{center}
		\includegraphics[width=0.4\linewidth]{../images/Tag}
	\end{center}
	\caption{\cite{www:Tag}}
\end{figure}

PME créée en 2005 à Toulouse, en collaboration avec le CNRS dans le cadre d'un programme de recherche et de développement.\\
La société est un acteur innovant sur le marché de l'alarme sans fil à travers ses marques Domotag et My-Fox.\\
Note obtenue dans le comparatif Alarmes de 60 millions de consommateurs : non noté.\\

\textbf{Définition du produit :}
\begin{itemize}
\item Centrale qui se relie à une ligne RTC pour délivrer ses alertes par le réseau IP sous forme d'emails.
\end{itemize}

\textbf{Différences remarquées en comparaison avec le système E-protect :}
\begin{itemize}
\item Télécommande 4 boutons pour piloter le système à distance\\
\end{itemize}

\textbf{mhouse}\\
\begin{figure}[h!]
	\begin{center}
		\includegraphics[width=0.4\linewidth]{../images/mhouse}
	\end{center}
	\caption{\cite{www:mhouse}}
\end{figure}

Filiale du groupe Italien Nice, mhouse propose des systèmes d'alarmes qui se distingue par leur facilité d'installation et d'utilisation grâce à des interfaces intuitives.\\
Note obtenue dans le comparatif Alarmes de 60 millions de consommateurs : non noté.\\

\textbf{Définition du produit :}
\begin{itemize}
\item Centrale d'alarme avec sirène, synthèse vocale et ransmetteur RTC/GSM intégrés.
\end{itemize}

\textbf{Différences remarquées en comparaison avec le système E-protect :}
\begin{itemize}
\item Relié au secteur avec Batterie intégrée 
\item Communication radio sur les canaux 433 et 868MHz
\item Detecteur d'ouverture
\item Commande vocale
\item Badge transpondeur MAB1\\
\end{itemize}

\textbf{Daitem}\\
\begin{figure}[h!]
	\begin{center}
		\includegraphics[width=0.4\linewidth]{../images/daitem}
	\end{center}
	\caption{\cite{www:daitem}}
\end{figure}

Autre marque de la société Atal, marque d'alarme haute gamme, onéreuse, vendu exclusivement auprès d'installateurs. Pas de pack proposé au particulier.

\textbf{Somfy}\\
\begin{figure}[h!]
	\begin{center}
		\includegraphics[width=0.4\linewidth]{../images/Somfy}
	\end{center}
	\caption{\cite{www:Somfy}}
\end{figure}

Créée dans les années 60 et aujourd'hui cotée en bourse, Somfy est le leader mondial des systèmes d'automatismes d'ouvertures et de fermetures des stores, volets roulant, etc...\\
Somfy propose aussi des système d'alarme NFA2P.\\
Note obtenue dans le comparatif Alarmes de 60 millions de consommateurs : 9.5/20.\\

\textbf{Définition du produit :}
\begin{itemize}
\item Centrale d'alarme sans fil.
\end{itemize}

\textbf{Différences remarquées en comparaison avec le système E-protect :}
\begin{itemize}
\item Sirène intégrée
\item Ne dispose pas nativement d'éléments de communication
\item Module RTC ou GSM en option
\item Detecteur d'ouverture à contact magnétique
\item Télécommande marche/arrêt\\
\end{itemize}


\section{Alarme à l'Internationnale :}

\textbf{Visionic}\\
\begin{figure}[h!]
	\begin{center}
		\includegraphics[width=0.4\linewidth]{../images/visionic}
	\end{center}
	\caption{\cite{www:visionic}}
\end{figure}

Société israélienne créée en 1973 faisant aujourd'hui partie du groupe Suisse TYCO, leader mondial des solutions de protections et de sécurité. Seulement disponible auprès de grossiste et des revendeurs spécialisés.

\textbf{Définition du produit :}
\begin{itemize}
\item Centrale d'alarme sans fil.
\end{itemize}

\textbf{Différences remarquées en comparaison avec le système E-protect :}
\begin{itemize}
\item Central doté de la technologie PowerG qui simplifie les installations, les éconoies d'énergie et une liaison sans fil fiable 
\item Cryptage de type AES\\
\end{itemize}

\textbf{Jablotron}\\
\begin{figure}[h!]
	\begin{center}
		\includegraphics[width=0.4\linewidth]{../images/jablotron}
	\end{center}
	\caption{\cite{www:jablotron}}
\end{figure}

Société Tchèque peu commercialisé en France.

\textbf{Définition du produit :}
\begin{itemize}
\item Centrale d'alarme sans fil.
\end{itemize}

\textbf{Différences remarquées en comparaison avec le système E-protect :}
\begin{itemize}
\item Central qui intègre divers protocole : RTC, GPRS, LAN\\
\end{itemize}

\textbf{Honeywell}\\
\begin{figure}[h!]
	\begin{center}
		\includegraphics[width=0.4\linewidth]{../images/honeywell}
	\end{center}
	\caption{\cite{www:honeywell}}
\end{figure}

Créée dans les années 90, Honeywell conçoit et commercialise aux travers de ses marques Domonial et Ademco des systèmes d'alarme ant-intrusion.\\
Commercialisation très discrète en France.\\

\textbf{Chuango}\\
\begin{figure}[h!]
	\begin{center}
		\includegraphics[width=0.4\linewidth]{../images/chuango}
	\end{center}
	\caption{\cite{www:chuango}}
\end{figure}

Société Chinoise créée en 2001, Chuango est un des leaders mondial de l'alarme.\\
Produits peu commercialisés en France. 

\textbf{Définition du produit :}
\begin{itemize}
\item Centrale d'alarme GSM G5.
\end{itemize}

\textbf{Différences remarquées en comparaison avec le système E-protect :}
\begin{itemize}
\item Aucune fonctions supplémentaires.\\
\end{itemize}

\textbf{Hager}\\
\begin{figure}[h!]
	\begin{center}
		\includegraphics[width=0.4\linewidth]{../images/hager}
	\end{center}
	\caption{\cite{www:hager}}
\end{figure}

Hager Group a été créée en 1955 intervenant dans l'ingénierie électrique. La firme s'est depuis diversifié et a étendu son activité dans les alarmes pour domicile.

\textbf{Définition du produit :}
\begin{itemize}
\item Centrale d'alarme Logisty Serenity.\\
\end{itemize}

\textbf{Différences remarquées en comparaison avec le système E-protect :}
\begin{itemize}
\item Communication centrale/capteurs assurée par la technologies Twinband
\item Pas de fonction de communication téléphonique
\item L'ajout d'un module prevoit la transmission d'image\\
\end{itemize}

\textbf{Alarme paradoxe}\\
\begin{figure}[h!]
	\begin{center}
		\includegraphics[width=0.4\linewidth]{../images/alarme_paradoxe}
	\end{center}
	\caption{\cite{www:alarme_paradoxe}}
\end{figure}

Sociétée créée en 1989 qui bénificie d'un large réseau de distributeurs répartis dans une centaine de pays.

\textbf{Définition du produit :}
\begin{itemize}
\item Centrale d'alarme Magellan 6250.
\end{itemize}

\textbf{Différences remarquées en comparaison avec le système E-protect :}
\begin{itemize}
\item Central gérant les protocoles GPRS, GSM, RTC et d'un module double SIM
\item Communication sans fil fait sur les canaux 868 ou 433MHz\\
\end{itemize}


