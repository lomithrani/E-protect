\chapter{Resume :}

Le produit proposé est un système d'alarme résidentiel connecté, facile à installer et à configurer. Afin de proposer un système le moins chère possible, E-protect s'appuie sur l'elimination des intermédiaires et utilise un réseau meshe de capteurs intelligent, travaillant sur une portée moins importante jouant ainsi sur la connexion des capteurs entre eux.\\ L'installation des composants du système E-protect est plus simple qu'un système d'alarme filé. Nul besoin de percer les murs et de poser des conducteurs, ce qui réduit considérablement les frais d'installation.\\

Coté Hardware, une centrale d'alarme gère un parc de capteurs connectés suivant une topologie meshé (ZigBee/Bluetooth Low Energy/CSRmesh). L'alarme se désactive lorsque le client entre dans la maison (ou passe par l'application mobile). L'alarme est la plus discrète possible et interagi avec la partie software (Sigfox/Ethernet).\\

Une application mobile et un site web permettent de se tenir informé de l'etat de fonctionnement de l'alarme. De plus le client peut configurer l'alarme à distance (Ap- plication Android/iOS/Windows Phone, Javascript/PHP/C\#/SQL).\\

E-protect assure un contact permanent entre le client et le domicile.\\

Le système E-protect c'est :\\

\begin{itemize}
\item Application Smartphone permettant de surveiller, de modifier, d'activer ou désactiver le système à distance ;
\item Interface du système ;
\item Un site internet personnalisé au même fonction que l'application Smartphone ;
\item Un système Backup en cas de défaillance du réseau électrique ;
\item Un design de capteur innovant et modulable selon la volonté du client ;
\item Une consommation énergétique très faible, reposant sur un appel de puissance réduit du fait de l'utilisation d'un réseau meshé\\
\end{itemize}
